\documentclass[handout]{beamer}
%\usepackage[german]{babel}
%\usepackage[utf8x]{inputenc}
\usepackage{listings}
\lstset{
	frame=single,
	breaklines=true
}

\definecolor{verylightgray}{rgb}{0.9,0.9,0.9}

\usetheme{Madrid}
% Other valid themes
%   Antibes, Bergen, Berkeley, Berlin, Copenhagen
%   Darmstadt, Dresden, Frankfurt, Goettingen, Hannover
%   Ilmenau, JuanLesPins, Luebeck, Madrid, Malmoe
%   Marburg, Montpellier, PaloAlto, Pittsburgh, Rochester
%   Singapore, Szeged, Warsaw, boxes, default

%möglich: Antibes, Darmstadt, Frankfurt, Madrid, Montpellier, Singapore

\usecolortheme{dove}
% Other valid color schemes
%    albatross, beaver, beetle, crane, dolphin
%    dove, fly, lily, orchid, rose, seagull
%    seahorse, whale and the one and only wolverine

%möglich: albatross, beaver, dove, whale

\title[Reportbasierte CSP Erzeugung]{Reportbasierte CSP-Erzeugung}
\subtitle{Abschlussvortrag Projektgruppe}
\author[Klaassen]{Malte Klaassen}
%\institute[Kurzform]{Institut}
\date{2016-10-26}

\begin{document}

\begin{frame}%1
\titlepage
\end{frame}

\begin{frame}%2
	\frametitle{Inhaltsverzeichnis}
	\tableofcontents%[pausesections]
\end{frame}

\section{Ziel des Projekts}

\begin{frame}
\frametitle{Ziel}
\begin{itemize}
\item Tool zur Generierung von Content Security Policies aus Reports
\item Reports k\"onnen bspw. w\"ahrend Testl\"aufen erstellt werden
\item Kenntnisse der Seite (ausser bei Tests) sollen nicht n\"otig sein
\end{itemize}
\end{frame}

\section{Implementierung}

\begin{frame}
\frametitle{Implementierung - Module}
%\begin{itemize}
%\item 4 Module:
4 Module:
\begin{itemize}
\item Reverse Proxy zur Generierung/Sammlung von Reports
\item Reverse Proxy zum Test einer Policy
\item FastCGI Report Collector 
\item Generator
%\end{itemize}
\end{itemize}
\end{frame}

\begin{frame}
\frametitle{Implementierung - Reverse Proxies}
\framesubtitle{Generierung}
\begin{itemize}
\item Basiert auf Docker-Container \colorbox{verylightgray}{\lstinline[basicstyle=\ttfamily\color{black}]|FROM nginx|}
\item Virtual Server mit
\begin{itemize}
\item \colorbox{verylightgray}{\lstinline[basicstyle=\ttfamily\color{black}]|fastcgi\_pass 127.0.0.1:9000;|} f\"ur \colorbox{verylightgray}{\lstinline[basicstyle=\ttfamily\color{black}]|/csprg\_collector.php|}
\item \colorbox{verylightgray}{\lstinline[basicstyle=\ttfamily\color{black}]|proxy\_pass \$SERVER;|} f\"ur \colorbox{verylightgray}{\lstinline[basicstyle=\ttfamily\color{black}]|/|}
\item \colorbox{verylightgray}{\lstinline[basicstyle=\ttfamily\color{black}]|add\_header "Content-Security-Policy-Report-Only"|}\newline\colorbox{verylightgray}{\lstinline[basicstyle=\ttfamily\color{black}]|"default-src 'none'; report-uri /csprg\_collector.php;"|}
\end{itemize}
\item Port Forwarding vom Host auf Port 8080
\end{itemize}
\end{frame}

\begin{frame}
\frametitle{Implementierung - Reverse Proxies}
\framesubtitle{Testing/Production}
\begin{itemize}
\item Basiert auf Docker-Container \colorbox{verylightgray}{\lstinline[basicstyle=\ttfamily\color{black}]|FROM nginx|}
\item Virtual Server mit
\begin{itemize}
\item \colorbox{verylightgray}{\lstinline[basicstyle=\ttfamily\color{black}]|fastcgi\_pass 127.0.0.1:9000;|} f\"ur \colorbox{verylightgray}{\lstinline[basicstyle=\ttfamily\color{black}]|/csprg\_collector2.php|}
\item \colorbox{verylightgray}{\lstinline[basicstyle=\ttfamily\color{black}]|proxy\_pass \$SERVER;|} f\"ur \colorbox{verylightgray}{\lstinline[basicstyle=\ttfamily\color{black}]|/|}
\item \colorbox{verylightgray}{\lstinline[basicstyle=\ttfamily\color{black}]|add\_header "Content-Security-Policy" \$CSP|}
\end{itemize}
\item Port Forwarding vom Host auf Port 80
\end{itemize}
\end{frame}

\begin{frame}
\frametitle{Implementierung - Collection}
\begin{itemize}
\item Basiert auf Docker-Container \colorbox{verylightgray}{\lstinline[basicstyle=\ttfamily\color{black}]|FROM php:fpm|}
\item Erh\"alt Reports von Proxies und gibt diese an Generator weiter
\item Nutzt Datei in Shared Memory f\"ur IPC
\end{itemize}
\end{frame}

\begin{frame}[fragile]
\frametitle{Reportstruktur - Chrome}
\lstset{
	alsoletter=-,
	morekeywords={blocked-uri,effective-directive},
	keywordstyle=\color{red},
	backgroundcolor=\color{verylightgray},
	basicstyle=\ttfamily\color{black}
}
\begin{lstlisting}
{"csp-report":
  { "document-uri":"http://localhost:8080/wiki/Main_Page"
  , "referrer":""
  , "violated-directive":"default-src 'none'"
  , "effective-directive":"script-src"
  , "original-policy":"default-src 'none'; report-uri /csprg_collector.php"
  , "blocked-uri":"inline"
  , "status-code":200
  }
}
\end{lstlisting}
\end{frame}

\begin{frame}[fragile]
\frametitle{Implementierung - Generierung}
\lstset{
	alsoletter=-,
	morekeywords={blocked-uri,effective-directive},
	keywordstyle=\color{red},
	backgroundcolor=\color{verylightgray},
	basicstyle=\ttfamily\color{black}
}
\begin{lstlisting}
Input: Reports, Blacklist, Whitelist
====================================
Reduce reports to 
	( report[effective-directive]
	, report[blocked-uri])
Check for and insert keywords
Check for blacklist matches
Starting with the whitelist fold reports into a Map (effective-directive -> List blocked-uri)
Flatten the Map into a policy
Return policy
\end{lstlisting}
%\begin{itemize}
%\item Reduziere Reports auf \colorbox{verylightgray}{\lstinline[basicstyle=\ttfamily\color{black}]|(report[effective-directive], report[blocked-uri])|}
%\item
%\end{itemize}
\end{frame}

\section{Demo}
\begin{frame}
Demo!
\end{frame}


\section{Hindernisse und Probleme}


%\subsection{Inline Skripte}
\begin{frame}[c]
\frametitle{Inline Skripte}
\begin{itemize}
\item \colorbox{verylightgray}{\lstinline[basicstyle=\ttfamily\color{black}]|script-src 'unsafe-inline'|} sollte unter allen Umst\"anden vermieden werden
\item Alternativen sind Noncen oder Hashes
\item Noncen erforden Eingriff in \"ubertragenes HTML
\item Hashes sind nicht in Reports enthalten
\end{itemize}
$\implies$ k\"onnen keine Policies f\"ur Seiten mit Inline-Skripten erstellen\newline
\end{frame}

%\subsection{Inkompatible Reports}
%\begin{frame}[c]
%\frametitle{Inkompatible Reports}
%\begin{itemize}
%
%\end{itemize}
%\end{frame}

%\subsection{Reportquellen}
\begin{frame}
\frametitle{Reportquellen}
\begin{itemize}
\item Bei Generierung aus Reports gesendet von normalen Seitenbesuchern: Angreifer kann beliebige Reports generieren
\item Reports m\"ussten also verifiziert werden \newline$\implies$ braucht exakte Kenntnisse der Seite \newline$\implies$ k\"onnten Policy direkt erstellen
\item K\"onnen also nur Reports aus vertrauensw\"urdigen Quellen verwenden
\end{itemize}
$\rightarrow$ L\"osen dies hier durch eigenen Virtual Server (+Firewall)
\end{frame}


%\subsection{Persistent XSS}
\begin{frame}[c]
\frametitle{Persistent XSS}
\begin{itemize}
\item Wird Generierung auf von Persistent XSS bereits betroffenen Seiten durchgef\"uhrt so wird dies durch CSP nicht verhindert
\item Generierung muss auf sauberen Daten erfolgen
\begin{itemize}
\item Testserver/-datens\"atze $\implies$ Aufwand/nicht immer verf\"ugbar
\item Generierung auf Seitensubset $\implies$ erfasst m\"oglicherweise nicht alle genutzten Ressourcen
\end{itemize}
\end{itemize}
\end{frame}

%\subsection{\"Anderungen in 3$^{rd}$ Partey Ressourcen}
%\begin{frame}[c]
%\frametitle{\"Anderungen in 3$^{rd}$ Partey Abh\"angigkeiten}
%\begin{itemize}
%\item \"Anderungen an eingebunden Fremd-Skripten f\"uhren gegebenenfalls zu zus\"atzlichen ben\"otigten Sourcen $\implies$ neuer Policy
%\end{itemize}
%\end{frame}

\begin{frame}
\frametitle{Weitere Probleme}
\begin{itemize}
\item \"Anderungen in 3$^{rd}$ Partey Ressourcen
\item Reportinkompatiblit\"aten
\item Tools ohne CSP-Support
\end{itemize}
\end{frame}

\section{}

\begin{frame}[c]
\frametitle{Zusammenfassung}
\begin{itemize}
\item Haben Werkzeug zur Reportbasierten Erzeugung von Content Security Policies
\item F\"ur Grossteil von Websiten grundlegend reportbasierte Generierung jedoch nicht m\"oglich
\item Komplexere (reportbasierte) Generierungsmethoden h\"aufig anf\"allig
\end{itemize}
$\implies$ Keine wirkliche Alternative zur Generierung von Hand/durch direkte Analyse der Seite
\end{frame}

\begin{frame}

\end{frame}

\begin{thebibliography}{}
\bibitem[2]{2}
GitHub \url{https://github.com/malteklaassen/pg_ss2016}
\bibitem[1]{1}
Content Security Policy Level 2; W3C Candidate Recommendation \url{https://www.w3.org/TR/CSP2/} (Abgerufen 2016-06-15)
\end{thebibliography}



\end{document}

%%% OLD %%%

%\subsection{sub}

%\begin{frame}[c]%24
%\frametitle{Vielen Dank für Ihre Aufmerksamkeit}
%\framesubtitle{Untertitel}
%\begin{itemize}
%\item Ein Item
%\end{itemize}
%\end{frame}
